\documentclass[a4paper,12pt]{article}
\usepackage[english,ukrainian,russian]{babel}
\linespread{1}
\usepackage{ucs}
\usepackage[utf8]{inputenc}
\usepackage[T2A]{fontenc}
\usepackage[paper=portrait,pagesize]{typearea}
\usepackage{amsmath}
\usepackage{bigints}
\usepackage{amsfonts}
\usepackage{graphicx}
\usepackage{amssymb}
\usepackage{cancel}
\usepackage{gensymb}
\usepackage{multirow}
\usepackage{rotate} 
\usepackage{pdflscape}
\usepackage{bigstrut}
\usepackage[pageanchor]{hyperref}
\usepackage{chngpage}
\usepackage{fancybox,fancyhdr}
\newcommand\tab[1][1cm]{\hspace*{#1}}
\newcommand{\arch}{\textrm{arcch}}
\newcommand{\arsh}{\textrm{arcsh}}
\newcommand{\dint}{\displaystyle\int}
\newcommand{\dsum}{\displaystyle\sum}
\newcommand{\RomanNumeralCaps}[1]{\MakeUppercase{\romannumeral #1}}
\usepackage[left=15mm, top=20mm, right=15mm, bottom=15mm, nofoot]{geometry}


\begin{document}
    \pagestyle{fancy}
    \fancyhead{}
    \fancyhead[R]{ФІ-12 Завалій Олександр}
    \begin{center}
        \large{\textbf{Міністерство освіти і науки України\\
                Національний технічний університет України\\
                «Київський політехнічний інститут імені Ігоря Сікорського»\\
                Навчально-науковий Фізико-технічний інститут}}\\
        \hfill \break \hfill \break \hfill\break \hfill \break \hfill \break \hfill \break \hfill \break
        \hfill \break \hfill \break \hfill \break
        \begin{center}
            \normalsize{\textbf{Моделювання природничих, \\ економічних та соціальних процесів\\
            Практичне завдання №4}}
        \end{center}
    \end{center}
    \hfill \break \hfill \break \hfill \break \hfill \break \hfill \break \hfill \break \hfill \break
    \hfill \break \hfill \break \hfill \break \hfill \break 
    \begin{flushright}
        \large{ \hspace{35pt} Виконав:\\
            студент групи ФI-12\\
            Завалій Олександр\\} 
        %\large{ \hspace{35pt} Перевірив:\\
        %} 
    \end{flushright}
    \hfill \break \hfill \break 
    \hfill \break \hfill \break \hfill \break \hfill \break \hfill \break \hfill \break \hfill \break
    \hfill \break
    \begin{center} \textbf{Київ-2025} \end{center}
    \thispagestyle{empty}

\newpage
    \begin{center}
        \section*{\bfseries{Практичне завдання №4. \\
        Аналіз чутливості параметрів стохастичних моделей
        }}
    \end{center}
    \textbf{Мета:} \\
    Розібратися в потенціалі використання штучного інтелекту для аналізу та покращення існуючих моделей різних процесів. \\
    \textbf{Завдання:} \\
    Оберіть модель для аналізу із пройдених у курсі детермінованих та стохастичних моделей НА ВАШ РОЗСУД, але окрім моделі Мальтуса. Приклади моделей: модель економічного замовлення (EOQ), моделі соціальної динаміки, тощо.
    Проаналізуйте наступні питання:
    \begin{enumerate}
        \item Способи застосування:
        \begin{enumerate}
            \item Використовуючи знання про модель, запропонуйте конкретні способи застосування нейронних мереж або методів ШІ. \\
            Наприклад, прогнозування динаміки популяції на основі історичних даних у моделі Мальтусa; виявлення прихованих факторів впливу на популяцію, тощо.
        \end{enumerate}
        
        \item Аналіз перваг та недоліків:
        \begin{enumerate}
            \item Опишіть, які переваги може надати використання ШІ порівняно з класичними детермінованими та стохастичними моделями в контексті обраної моделі.
            \item Проаналізуйте потенційні недоліки та обмеження застосування ШІ в контексті обраної моделі.
        \end{enumerate}
        
        \item Гібридний підхід:
        \begin{enumerate}
            \item  Опишіть, як можна поєднати диференціальні рівняння з нейронними мережами. \\
            Наприклад, застосування Neural ODE для створення більш точної та комплексної моделі популяційного зростання.
            \item Поясніть, які аспекти моделі можна краще описати за допомогою такого гібридного підходу.
        \end{enumerate}
    \end{enumerate}

\newpage
\noindent
    Розглянемо \textbf{моделі соціальної динаміки}. \\
    \textbf{Способи застосування ШІ}
    \begin{enumerate}
        \item Прогнозування соціальної динаміки \\
        Ніби банально, але сенс в цьому є. Можемо використовувати нейронні мережі для аналізу історичних даних та прогнозування змін у соціальній поведінці. Наприклад, рекурентні нейромережі (RNN, LSTM) можуть виявляти патерни у зміні суспільних настроїв, міграційних потоків тощо.
        
        \item Виявлення прихованих факторів впливу \\
        Використання методів глибокого навчання для аналізу соціальних мереж і визначення ключових факторів, які впливають на зміну громадської думки чи поведінки.
        
        \item Аналіз муйбутніх подій \\
        Генеративні моделі (наприклад, GAN або VAEs) можуть допомогти створювати альтернативні сценарії майбутнього, змінюючи параметри моделей соціальної динаміки.
    \end{enumerate}
    \textbf{Переваги та недоліки використання ШІ}
    \begin{enumerate}
        \item Переваги:
        \begin{enumerate}
            \item Адаптивність до складних соціальних процесів \\
            Класичні стохастичні та детерміновані моделі зазвичай мають в основі визначені рівняннях і параметрах, що обмежує їх здатність до адаптації.
            ШІ, особливо глибокі нейронні мережі, може динамічно оновлювати параметри на основі поточних даних. Це дозволяє краще адаптувати модель до реальних змін у соціальній динаміці.
            
            \item Виявлення нелінійних та прихованих зв’язків \\
            ШІ здатний виявляти складні нелінійні взаємозв’язки між соціальними факторами.
            
            \item Обробка великих обсягів даних \\
            Традиційні підходи зазвичай вимагають ручного підбору параметрів та можуть бути обмеженими при аналізі великого об'єму інформації.
            ШІ здатний ефективно працювати з великими даними, аналізуючи соціальні мережі, економічні показники, демографічні тенденції тощо.
            
            \item Можливість прогнозування на основі реальних даних \\
            Стохастичні моделі працюють на основі ймовірнісних припущень, що не завжди дозволяє точно передбачати майбутній розвиток подій.
            Нейромережі, особливо рекурентні (RNN, LSTM), можуть прогнозувати соціальну динаміку, навчаючись на історичних даних.
        \end{enumerate}
        \item Недоліки:
        \begin{enumerate}
            \item Проблема інтерпретованості \\
            Детерміновані моделі надають зрозумілі математичні рівняння, які можна легко аналізувати.

            \item Залежність від якісних та повних даних \\
            Класичні соціальні моделі можуть працювати навіть за умов обмежених даних.
            Нейромережі потребують великих, репрезентативних і якісних наборів даних.
        
            \item Нестабільність та ризик перенавчання \\
            Стохастичні моделі мають чітко визначені статистичні параметри.
            Нейромережі можуть бути чутливими до шуму в даних і демонструвати перенавчання (overfitting).
        
            \item Висока обчислювальна складність \\
            Глибокі нейромережі потребують значних ресурсів для навчання.
        
            \item Проблеми узагальнення на різні соціальні класи \\
            Класичні моделі часто базуються на загальних теоретичних принципах.
            ШІ може добре працювати у певному регіоні або часовому періоді, але показувати погані результати в інших сферах.
        \end{enumerate}
    \end{enumerate}
    \textbf{Гібридний підхід} \\
    Гібридний підхід поєднує класичні моделі диференціальних рівнянь із глибоким навчанням. 
    Водночас дозволяючи адаптувати параметри на основі даних.
    Можемо визначити таки способи застосування:
    \begin{enumerate}
        \item Поєднання епідеміологічних моделей (SIR) із нейромережами може допомогти краще зрозуміти динаміку поширення новин або дезінформації в соціальних мережах.
        \item З використанням Neural ODE можемо покращити прогнозування, оновлюючи параметри системи на основі поточних даних.
        \item Використання гібридних підходів для моделювання впливу законодавчих ініціатив на громадську думку.
    \end{enumerate}
    \textbf{Висновки} \\
    Використання ШІ у соціальних моделях є доцільним у випадках, коли потрібно працювати з великими обсягами даних і 
    знаходити складні патерни. Або якщо є необхідність у швидкому адаптивному прогнозуванні.
    Також це може допомогти виявляти приховані фактори, які не завжди враховуються у класичних моделях. \\ 
    Проте для забезпечення прозорості та інтерпретованості результатів найбільш ефективним підходом є поєднання класичних стохастичних моделей із методами глибокого навчання.





\end{document}